\begin{abstract}
The Oblique Flying Wing configuration (OFW) has historically been a difficult configuration to control.
Lacking empennage control surfaces, control authority is limited and has led to many compromises in aerodynamic efficiency.
Taking lessons learned from legacy flight test programs such as the X-53 Active Aeroelastic Wing (AAW) program, the benefits of including leading edge control surfaces for maneuvers is evident.
With maturing compliant structures technology, integration of a compliant leading edge into the early stage design process of an OFW can aid in improving the controllability and aerodynamic performance.
Utilizing parametric geometry tools, gradient based optimization, and RANS CFD, the over-determined rigid trim problem can be solved.
Two OFW configurations, with and without leading edge compliant control surfaces, are compared at three flight conditions.
Evaluating the ability of each configuration to satisfy the trim constraints and produce the required yaw moments will determine the effectiveness of including compliant leading edge control surfaces on an OFW.
\end{abstract}

\section{Nomenclature}

{\renewcommand\arraystretch{1.0}
\noindent\begin{longtable*}{@{}l @{\quad=\quad} l@{}}
$C_L$ & Lift Coefficient\\
$C_D$ & Drag Coefficient\\
$C_S$ & Side Force Coefficient\\
$C_l$ & Rolling Moment Coefficient\\
$C_m$ & Pitching Moment Coefficient\\
$C_n$ & Yawing Moment Coefficient\\
$c$ & Chord (m)\\
$\overline{c}$ & Mean Aerodynamic Chord (m)\\
$b$ & Span (m)\\
$\Lambda$ & Slew Angle ($^\circ$) \\
$\alpha$ & Angle of attack ($^\circ$)\\
$\delta_i$ & Control Surface Deflection ($^\circ$)\\
\end{longtable*}}
